\input{preambuloSimple.tex}  % Configuración del documento

%----------------------------------------------------------------------------------------
%	TÍTULO Y DATOS DE LOS ALUMNOS
%----------------------------------------------------------------------------------------

\title{	
	\normalfont \normalsize 
	\textsc{\textbf{Fundamentos de Redes (2017-2018)} \\ Doble Grado en Ingeniería Informática y Matemáticas \\ Universidad de Granada} \\ [25pt] 
	\horrule{0.5pt} \\[0.4cm]
	\huge Definición e implementación de un \\ protocolo de aplicación \\ 
	\horrule{2pt} \\[0.5cm] 
}

\author{Simón López Vico \\ Ana María Peña Arnedo \\ Alberto Jesús Durán López} 
\date{\normalsize\today}

%----------------------------------------------------------------------------------------
% DOCUMENTOg
%----------------------------------------------------------------------------------------

\begin{document}
	\maketitle       % título
	\newpage 
	\tableofcontents % índice
	\newpage
	
	

	
	
\section{Introducción}
	
	
Esta práctica consiste en la implementación de un protocolo de aplicación que consta
de un servidor y dos clientes. El juego se basa en el famoso "Pilla-Pilla"   donde 
un jugador 'X' tendrá que alcanzar a otro 'Y'. Para ello, ambos se conectarán con un nombre de usuario correcto almacenado en el servidor. Por tanto, los movimientos que ambos hagan se enviarán a través del servidor hacia el otro cliente.
		
	
	
	
\section{Diagrama de estados del servidor}

\begin{figure}[h]
	\centering
	\includegraphics[width=.8\textwidth]{img/1}
	\caption{Diagrama de estados del servidor}
\end{figure}
	

Empezamos en el estado \textit{START}. El cliente abre la conexión hacia el servidor. Por tanto, pasará al estado \textit{IDENTIFICADO} si introduce un usuario correcto ó se quedará en el estado \textit{NO IDENTIFICADO} hasta que se autentifique correctamente. Una vez conectados ambos clientes, podrán realizar movimientos y pasar al estado \textit{DIFUNDIR MOVIMIENTO} un número indefinido de veces o pasar al estado \textit{END} si el juego se ha acabado. Cabe destacar que cada estado posee su propia comprobación de errores.
	












\newpage


\section{Tabla de estados del servidor}

\begin{figure}[h]
	\centering
	\includegraphics[width=.8\textwidth]{img/3}
	\caption{Mensajes que intervienen en el servidor}
\end{figure}


















\section{Tabla de estados del cliente}	

\begin{figure}[h]
	\centering
	\includegraphics[width=.8\textwidth]{img/2}
	\caption{Mensajes que intervienen en el cliente}
\end{figure}













\end{document}